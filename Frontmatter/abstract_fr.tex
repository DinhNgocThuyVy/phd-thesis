\noindent Cette thèse examine l'application de la Factorisation Matricielle Positive (PMF) pour l'attribution des sources de pollution atmosphérique. Nous développons des techniques d'optimisation avancées et des cadres de validation qui améliorent la précision et la fiabilité de la PMF dans l'identification des sources de pollution dans les environnements urbains européens.

Notre recherche démontre que des protocoles améliorés de configuration et de validation du modèle peuvent réduire significativement les incertitudes dans les estimations de contribution des sources, fournissant aux décideurs politiques des données plus fiables pour des interventions ciblées.

\vspace{0.5cm}

\noindent \textbf{Mots-clés:} mot-clé1, mot-clé2, mot-clé3, mot-clé4
